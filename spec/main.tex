
\documentclass{scrreprt}
\usepackage{listings}
\usepackage{underscore}
\usepackage[bookmarks=true]{hyperref}
\usepackage[utf8]{inputenc}
\usepackage[english]{babel}
\usepackage[fleqn]{amsmath}
\usepackage{syntax}

\hypersetup{
    pdftitle={Software Requirement Specification},    % title
    pdfauthor={Jean-Philippe Eisenbarth},                     % author
    pdfsubject={TeX and LaTeX},                        % subject of the document
    pdfkeywords={TeX, LaTeX, graphics, images}, % list of keywords
    colorlinks=true,       % false: boxed links; true: colored links
    linkcolor=blue,       % color of internal links
    citecolor=black,       % color of links to bibliography
    filecolor=black,        % color of file links
    urlcolor=purple,        % color of external links
    linktoc=page            % only page is linked
}%

\def\myversion{1.0 }
\date{}
%\title
\usepackage{hyperref}
\begin{document}

\begin{flushright}
    \rule{16cm}{5pt}\vskip1cm
    \begin{bfseries}
        \Huge{LANGUAGE SPECIFICATION}\\
        \vspace{1.9cm}
        for\\
        \vspace{1.9cm}
        BNF Meta Language (BNF-ML)\\
        \vspace{1.9cm}
        \LARGE{Version \myversion approved}\\
        \vspace{1.9cm}
        Prepared by Andre Tabourian\\
        \vspace{1.9cm}
        \today\\
    \end{bfseries}
\end{flushright}

\tableofcontents


\chapter*{Revision History}

\begin{center}
    \begin{tabular}{|c|c|c|c|}
        \hline
	    Name & Date & Reason For Changes & Version\\
        \hline
	    21 & 22 & 23 & 24\\
        \hline
	    31 & 32 & 33 & 34\\
        \hline
    \end{tabular}
\end{center}

\chapter{ABOUT}

BNFML is a programming language based on the Backus Naur Form (BNF) used to lex and parse the BNF language itself.

\chapter{Language Specification}

\section{Grammar}

    The following, is the BNF specification of BNFML.

    \begin{grammar}

        <BNF> ::= <List of Bindings> | <empty>

        <List of Bindings> ::= <List of Bindings> <Binding> | <Binding>

        <Binding> ::= <NonTerminal> `::=' <OrExpression>

        <OrExpression> ::= <OrExpression> `|' <AndExpression> | <AndExpression>

        <AndExpression> ::= <AndExpression> <Expression> | <Expression>

        <Expression> ::= <NonTerminal> | <Terminal>
    
        <NonTerminal> ::= `<.+>'
        
        <Terminal> ::= `.+'

        <empty> ::= `'

    \end{grammar}
    
    Please note that BNFML is self-descriptive, as such the only specification needed to describe BNFML is BNFML itself. What this means is that the specification you see above is BNFML,
    and it describes itself.

\section{Explanation of the Spec}

The following is an explanation of the above specification:

\begin{itemize}
    \item A \emph{BNF} consists of a list of \emph{Bindings} or nothing (`').
    \item A \emph{Binding} consists of a \emph{NonTerminal} followed by \emph{::=} and an \emph{OrExpression}
    \item An \emph{OrExpression} consists of an \emph{AndExpression} or another \emph{OrExpression} followed by a pipe \emph{$|$} and an \emph{AndExpression}
    \item An \emph{AndExpression} consists of an \emph{Expression} or another \emph{AndExpression} followed by an \emph{Expression}
    \item An \emph{Expression} is nothing more than a \emph{NonTerminal} or a \emph{Terminal}
    \item A \emph{NonTerminal} is denoted by angle brackets $< >$ containing any character.
    \item A \emph{Terminal} is denoted by single quotes containing any regular expressions. Note that that to insert single quotes in a terminal, they must be escaped with \textbackslash
\end{itemize}

\section{Whitespace}

This section deals with whitespace characters:\\

\underline{New Lines:}\\

- Two Bindings are separated by at least one new line.

- New lines only separate bindings and nothing else.\\

\underline{Spaces and Tabs:}\\

- Whitespace does not matter except within a Terminal.

- `1' `2' and `1'`2' are interpreted in the same way.

- $<a>\ \ ::= \ <b>$ and $<a>::<b>$ are interpreted in the same way.


\chapter{Implementation}

\section{Lexer Implementation}

\section{Parser Implementation}

\end{document}